\documentclass [a4paper,12pt] {article}
\usepackage [T1] {fontenc}
\usepackage [utf8] {inputenc}
\usepackage [french] {babel}

%============================================================

\usepackage {amsmath}	% align*
\usepackage {amssymb}	% mathbb
\usepackage {mathrsfs}	% mathscr
\usepackage {amsthm}	% newtheorem
%\usepackage {ebproof}	% prooftree

\usepackage{mathpartir}

%============================================================

\setlength \parindent {0pt}

%============================================================

\usepackage {enumitem}
\setlist [itemize] {leftmargin=*}

%============================================================

\newtheoremstyle {definition}
	{}		% space above
	{}		% space below
	{}		% body font
	{}		% indent
	{\bf}	% head font
	{\\}	% head punctuation
	{ }		% head space
	{}		% head spec

\newenvironment {preuve} 
	{\begin {proof} ~\\} 
	{\end {proof}}

\theoremstyle {definition}

\newtheorem {definition} {Définition} [section]
\newtheorem {lemme} {Lemme} [section]
\newtheorem {theoreme} {Théorème} [section]

%============================================================

\title {
	Recherche de fonctions par types \\[0.5em]
	\large Un peu de théorie
}
\date {}

%============================================================

\newcommand {\interval} [2] {[\![#1\,;#2]\!]}

\newcommand {\ssi} {\textit {ssi}}

\newcommand {\V} {\mathscr V}
\newcommand {\C} {\mathscr C}
\newcommand {\T} {\mathscr T}
\newcommand {\E} {\mathscr E}

\newcommand {\qeq} {\stackrel ? =}
\newcommand {\Eeq} {\stackrel \E =}

%============================================================
%============================================================

\begin {document}

\maketitle

%============================================================

\section {Signature}

%%%%\inferrule{toto}{totoi}

\begin {definition} [signature]
	Une signature est un ensemble de symboles de fonctions d'arité finie. \\
	On signifie par $(f : n) \in \Sigma$ le fait que le symbole $f$ de fonction d'arité $n$ appartienne à la signature $\Sigma$.
\end {definition}

Dans toute la suite, on se donne un ensemble dénombrable $\V$ de symboles de variables ainsi qu'une signature $\Sigma$ telle que $\V \cap \Sigma = \emptyset$, $(unit : 0) \in \Sigma$, $(\cdot * \cdot : 2) \in \Sigma$ et $(\cdot \rightarrow \cdot \, : 2) \in \Sigma$.

%============================================================

\section {Types}

\begin {definition} [type]
	L'ensemble des types, noté $T$, est défini inductivement par :
	\begin {center}
          \inferrule{0}{\V \subseteq T}  %% voilà un exemple
		% \begin {prooftree}
		% 	\Infer 0 {\V \subseteq T}
		% \end {prooftree}
		% ~~~
		% \begin {prooftree}
		% 	\Hypo {(f : n) \in  \Sigma}
		% 	\Hypo {\forall i \in \interval 1 n,\ \tau_i \in T}
		% 	\Infer 2 {f (\tau_1, \dots, \tau_n) \in T}
		% \end {prooftree}
	\end {center}
\end {definition}



\begin {definition} [type-flèche]
	Un type-flèche (ou une flèche) est un type de la forme : $\tau_1 \rightarrow \tau_2$.
\end {definition}

\begin {definition} [mot sur $\mathbb N$]
	L'ensemble des mots sur $\mathbb N$ est défini inductivement par :
	\begin {center}
		% \begin {prooftree}
		% 	\Infer 0 {\Lambda \text{ mot vide sur } \mathbb N}
		% \end {prooftree}
		% ~~~
		% \begin {prooftree}
		% 	\Hypo {i \in \mathbb N}
		% 	\Hypo {p \text{ mot sur } \mathbb N}
		% 	\Infer 2 {i.p \text{ mot sur } \mathbb N}
		% \end {prooftree}
	\end {center}
\end {definition}

\begin {definition} [positions d'un type]
	L'ensemble des positions d'un type $\tau$, noté $Pos(\tau)$, est l'ensemble des mots sur $\mathbb N$ défini inductivement par :
	\begin {center}
		% \begin {prooftree}
		% 	\Infer 0 {\Lambda \in Pos(\tau)}
		% \end {prooftree}
		% ~~~
		% \begin {prooftree}
		% 	\Hypo {i \in \interval 1 n}
		% 	\Hypo {p \in Pos(\tau_i)}
		% 	\Infer 2 {i.p \in Pos(f (\tau_1, \dots, \tau_n))}
		% \end {prooftree}
	\end {center}
\end {definition}

\begin {definition} [sous-type]
	Le sous-type d'un type $\tau$ à la position $p$, noté $\tau [p]$, est défini par induction sur $p$ par :
	\begin {align*}
		\tau [\Lambda] &= \tau \\
		f (\tau_1, \dots, \tau_n) [i.q] &= \tau_i [q]
	\end {align*}
\end {definition}

\begin {definition} [affectation de types]
	Une affectation de types est une fonction de $\V$ dans $T$.
\end {definition}

\begin {definition} [substitution de types]
	Une substitution de type est un endomorphisme de $T$.
\end {definition}

\begin {definition} [substitution de types induite]
	Soit $\alpha$ une affectation de types. \\
	La substitution de types induite par $\alpha$, notée $\hat \alpha$, est la substitution de types dont la restriction à $\V$ est $\alpha$. \\
	Autrement dit, $\hat \alpha$ est définie inductivement par :
	\begin {align*}
		\hat \alpha (v) &= \alpha (v) \\
		\hat \alpha (f (\tau_1, \dots, \tau_n)) &= f (\hat \alpha (\tau_1), \dots, \hat \alpha (\tau_n))
	\end {align*}
\end {definition}

\begin {definition} [instance de type]
	Un type $\tau_1$ est une instance d'un type $\tau_2$, noté $\tau_2 \leqslant_T \tau_1$, s'il existe une substitution de types $\sigma$  telle que $\tau_1 = \sigma (\tau_2)$.
\end {definition}

%============================================================

\section {Théorie équationnelle}

\begin {definition} [axiome équationnel]
	Un axiome équationnel est un couple de types de la forme $\tau_1 \doteq \tau_2$.
\end {definition}

\begin {definition} [instance d'axiome équationnel]
	Une instance d'un axiome équationnel $\tau_1 \doteq \tau_2$ est un couple de types $(\tau_1', \tau_2')$ tels que $\tau_1'$ soit une instance de $\tau_1$ et $\tau_2'$ une instance de $\tau_2$.
\end {definition}

\begin {definition} [théorie équationnelle]
	Soit $\E$ un ensemble d'axiomes équationnels. \\
	La théorie équationnelle induite par $\E$, notée $\cdot \Eeq \cdot$, est la plus petite congruence sur $\Sigma$ contenant toutes les instances des axiomes équationnels de $\E$. \\
	Autrement dit, c'est la plus petite relation binaire satisfaisant les règles d'inférence :
	\begin {center}
		% \begin {prooftree}
		% 	\Hypo {\tau_1 \doteq \tau_2 \in \E}
		% 	\Infer 1 [($\Eeq$-ax)] {\hat \alpha (\tau_1) \Eeq \hat \alpha (\tau_2)}
		% \end {prooftree}
		% \\[1em]
		% \begin {prooftree}
		% 	\Infer 0 [($\Eeq$-refl)] {\tau \Eeq \tau}
		% \end {prooftree}
		% \\[1em]
		% \begin {prooftree}
		% 	\Hypo {\tau_1 \Eeq \tau_2}
		% 	\Hypo {\tau_2 \Eeq \tau_3}
		% 	\Infer 2 [($\Eeq$-trans)] {\tau_1 \Eeq \tau_3}
		% \end {prooftree}
		% \\[1em]
		% \begin {prooftree}
		% 	\Hypo {\tau_1 \Eeq \tau_2}
		% 	\Infer 1 [($\Eeq$-sym)] {\tau_2 \Eeq \tau_1}
		% \end {prooftree}
		% \\[1em]
		% \begin {prooftree}
		% 	\Hypo {(f : n) \in \Sigma}
		% 	\Hypo {\tau_i \Eeq \tau_i'}
		% 	\Infer 2 [($\Eeq$-congr)] {f (\tau_1, \dots, \tau_i, \dots, \tau_n) \Eeq f (\tau_1, \dots, \tau_i', \dots, \tau_n)}
		% \end {prooftree}
	\end {center}
\end {definition}

\begin {definition} [$\E$-équivalence]
	Soit $\E$ un ensemble d'axiomes équationnels. \\
	Deux types $\tau_1$ et $\tau_2$ sont $\E$-équivalents {\ssi} $\tau_1 \Eeq \tau_2$.
\end {definition}

\begin {definition} [$\E$-unifiabilité]
	Soit $\E$ un ensemble d'axiomes équationnels. \\
	Deux types $\tau_1$ et $\tau_2$ sont $\E$-unifiables, noté $\tau_1 \bowtie_\E \tau_2$, \ssi :
	\[ \exists \alpha \in \mathscr F (\mathscr V, T),\ \hat \alpha (\tau_1) \Eeq \hat \alpha (\tau_2) \]
\end {definition}

Dans toute la suite, on s'intéressa à l'ensemble $\E$ des axiomes équationnels (où $x$, $y$ et $z$ sont des variables) :
\begin {align*}
	x * (y * z) &\doteq (x * y) * z && \text {($*$-assoc)} \\
	x * y &\doteq y * x && \text {($*$-comm)} \\
	x * y \rightarrow z &\doteq x \rightarrow y \rightarrow z && \text {(curry)}
\end {align*}

On considérera également la théorie équationnelle $\cdot \Eeq \cdot$ induite par $\E$ ; on parlera d'équivalence pour l'$\E$-équivalence et d'unifiabilité pour l'$\E$-unifiabilité (notée $\cdot \bowtie \cdot$).

%============================================================

\section {Premier critère}

\begin {definition} [tête d'un type]
	La tête d'un type, notée $\uparrow \cdot$, est définie inductivement par :
	\begin {align*}
		\uparrow (\tau_1 \rightarrow \tau_2) &=\ \uparrow \tau_2 \\
		\uparrow \tau &= \tau
	\end {align*}
\end {definition}

\begin {lemme} \label {=E-tête}
	Si deux types sont équivalents, leurs têtes le sont aussi.
\end {lemme}

\begin {lemme} \label {tête-subst-tête}
	$\forall \tau \in T,\ \forall \alpha \in \mathscr F (\mathscr V, T),\ \uparrow \hat \alpha (\tau) =\ \uparrow \hat \alpha (\uparrow \tau)$
\end {lemme}

\begin {lemme} \label {tête-non-flèche}
	La tête d'un type n'est pas une flèche.
\end {lemme}

\begin {lemme} \label {cons-=E}
	$\forall (f_1 : n_1) \in \Sigma,\ \forall (f_1 : n_2) \in \Sigma,$ \\
	$\forall (\tau^1_i)_{i \in \interval 1 {n_1}} \in T^{n_1},\ \forall (\tau^2_i)_{i \in \interval 1 {n_2}} \in T^{n_2},$ \\
	$f_1 (\tau^1_1, \dots, \tau^1_{n_1}) = f_2 (\tau^2_1, \dots, \tau^2_{n_2}) \implies f_1 = f_2$
\end {lemme}

\begin {theoreme}
	$\forall \tau_1 \in T,\ \forall \tau_2 \in T,$ \\
	$\tau_1 \bowtie \tau_2 \wedge \left( \uparrow \tau_1 \right) [\Lambda] = f_1 \notin \V \wedge \left( \uparrow \tau_2 \right) [\Lambda] = f_2 \notin \V \implies f_1 = f_2$
\end {theoreme}

\begin {preuve}
	Par hypothèse, il existe une affectation de types $\alpha$ telle que :
	\[ \hat \alpha (\tau_1) \Eeq \hat \alpha (\tau_2) \]
	Par le lemme \ref {=E-tête}, les têtes sont équivalentes :
	\[ \uparrow \hat \alpha (\tau_1) \Eeq\ \uparrow \hat \alpha (\tau_2) \]
	Par le lemme \ref {tête-subst-tête}, on a donc :
	\[ \uparrow \hat \alpha (\uparrow \tau_1) \Eeq\ \uparrow \hat \alpha (\uparrow \tau_2) \]
	Par hypothèse, il existe $(f_1 : n_1)$ et $(f_2 : n_2)$ dans $\Sigma$ ainsi que $(\tau^1_i)_{i \in \interval 1 {n_1}}$ dans $T^{n_1}$ et $(\tau^2_i)_{i \in \interval 1 {n_2}}$ dans $T^{n_2}$ tels que :
	\begin {align*}
		\tau_1 &= f_1 (\tau^1_1, \dots \tau^1_{n_1}) \\
		\tau_2 &= f_2 (\tau^2_1, \dots \tau^2_{n_2})
	\end {align*}
	Par le lemme \ref {tête-non-flèche}, $f_1$ et $f_2$ sont différents de $\cdot \rightarrow \cdot$. \\
	Par definition de $\uparrow \cdot$ et $\hat \alpha$, il vient alors :
	\[ f_1 (\hat \alpha (\tau^1_1), \dots, \hat \alpha (\tau^1_{n_1})) \Eeq f_2 (\hat \alpha (\tau^2_1), \dots, \hat \alpha (\tau^2_{n_2})) \]
	Enfin, le lemme \ref {cons-=E} donne :
	\[ f_1 = f_2 \]
\end {preuve}

%============================================================

\section {Deuxième critère}

\begin {definition} [multiplicité de symbole de fonctions]
	La multiplicité d'un symbole de fonction $f$, notée $\mu_f$, est définie inductivement par :
	\begin {align*}
		\mu_f (\tau_1 \rightarrow \tau_2) &= \mu_f' (\tau_1) + \mu_f (\tau_2) \\
		\mu_f (\tau) &= 0 \\
		\mu_f' (\tau_1 * \tau_2) &= \mu_f' (\tau_1) + \mu_f' (\tau_2) \\
		\mu_f' (f (\tau_1, \dots, \tau_n)) &= 1 \\
		\mu_f' (\tau) &= 0
	\end {align*}
\end {definition}

\begin {definition} [$\V$-multiplicité]
	La $\V$-multiplicité est définie inductivement par :
	\begin {align*}
		\mu_\V (\tau_1 \rightarrow \tau_2) &= \mu_\V' (\tau_1) + \mu_\V (\tau_2) \\
		\mu_\V (\tau) &= 0 \\
		\mu_\V' (v) &= 1 \\
		\mu_\V' (\tau_1 * \tau_2) &= \mu_\V' (\tau_1) + \mu_\V' (\tau_2) \\
		\mu_\V' (\tau) &= 0
	\end {align*}
\end {definition}

\begin {definition} [type simpe]
	Un type $\tau$ est simple si sa $\V$-multiplicité est nulle.
\end {definition}

\begin {lemme} \label {mu-=E}
	Si deux type $\tau_1$ et $\tau_2$ sont équivalents, alors, pour tout symbole de fonction $f$, on a : $\mu_f (\tau_1) = \mu_f (\tau_2)$.
\end {lemme}

\begin {lemme} \label {mu-subst-simple}
	Si un type $\tau$ est simple, alors, pour tout symbole de fonction $f$ et toute affectation de types $\alpha$, on a : $\mu_f (\hat \alpha (\tau)) = \mu_f (\tau)$.
\end {lemme}

\begin {lemme} \label {mu-subst}
	La multiplicité de tout symbole de fonction $f$ dans un type est inférieure à celle de toute instance de ce type.
\end {lemme}

\begin {theoreme}
	Soit deux types $\tau_1$ et $\tau_2$. \\
	Si $\tau_1$ et $\tau_2$ sont unifiables et $\tau_1$ simple, alors la multiplicité de tout symbole de fonction dans $\tau_1$ est supérieure à celle dans $\tau_2$.
\end {theoreme}

\begin {preuve}
	Par hypothèse, il existe une affectation $\alpha$ telle que :
	\[ \hat \alpha (\tau_1) \Eeq \hat \alpha (\tau_2) \]
	Soit $f$ un symbole de fonction. \\
	Par le lemme \ref {mu-=E}, les multiplicités sont égales :
	\[ \mu_f (\hat \alpha (\tau_1)) = \mu_f (\hat \alpha (\tau_2)) \]
	Par le lemme \ref {mu-subst-simple}, la simplicité de $\tau_1$ apporte :
	\[ \mu_f (\hat \alpha (\tau_1)) = \mu_f (\tau_1) \]
	Par le lemme \ref {mu-subst}, on a par ailleurs :
	\[ \mu_f (\hat \alpha (\tau_2)) \geqslant \mu_f (\tau_2) \]
	Il vient donc le résultat attendu :
	\[ \mu_f (\tau_1) \geqslant \mu_f (\tau_2) \]
\end {preuve}

%============================================================

\section {Normalisation}

\begin {definition} [multi-ensemble]
	Un multi-ensemble sur un ensemble $X$ est une fonction de $X$ dans $\mathbb N$. \\
	L'ensemble des multi-ensembles sur $X$ est noté $X^\sharp$.
\end {definition}

\begin {definition} [domaine de multi-ensemble]
	Le domaine d'un multi-ensemble sur $X$ est l'ensemble des éléments de $X$ dont l'image est non-nulle. \\
	Si le domaine d'un multi-ensemble $m$ est fini, on note ce dernier $\{ x_1 : m (x_1) ; \dots ; x_n : m (x_n) \}$ où $(x_i)_{i \in \interval 1 n}$ est le domaine de $m$.
\end {definition}

\begin {definition} [somme de multi-ensembles]
	La somme de deux multi-ensemble sur $X$ $m_1$ et $m_2$, notée $m_1 +_m m_2$, est le multi-ensemble $x \mapsto m_1 (x) + m_2 (x)$.
\end {definition}

\begin {definition} [type normalisé]
	L'ensemble des types normalisés (ou $\nu$-types), noté $N$, est défini inductivement par :
	\begin {center}
		% \begin {prooftree}
		% 	\Infer 0 {\V \subseteq N}
		% \end {prooftree}
		% \\[1em]
		% \begin {prooftree}
		% 	\Hypo {(f : n) \in  \Sigma}
		% 	\Hypo {\forall i \in \interval 1 n,\ \nu_i \in N}
		% 	\Infer 2 {f (\nu_1, \dots, \nu_n) \in N}
		% \end {prooftree}
		% \\[1em]
		% \begin {prooftree}
		% 	\Hypo {m \in N^\sharp}
		% 	\Hypo {\nu \in N}
		% 	\Infer 2 {m \rightarrow_\nu \nu \in N}
		% \end {prooftree}
	\end {center}
\end {definition}

%============================================================

\end {document}

































