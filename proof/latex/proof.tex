\documentclass [a4paper,12pt] {article}
\usepackage [T1] {fontenc}
\usepackage [utf8] {inputenc}
\usepackage [french] {babel}

%============================================================

\usepackage {amsmath}	% align*
\usepackage {amssymb}	% mathbb
\usepackage {mathrsfs}	% mathscr

%============================================================

%\setlength \parindent {0pt}

%============================================================

\usepackage {enumitem}
\setlist [itemize] {leftmargin=*}

%============================================================

\usepackage {mathpartir}
\def \RightTirNameStyle {\textnormal}

%============================================================

\usepackage {amsthm}

\newtheoremstyle {definition}
	{}		% space above
	{}		% space below
	{}		% body font
	{}		% indent
	{\bf}	% head font
	{\\}	% head punctuation
	{ }		% head space
	{}		% head spec

\newenvironment {preuve} 
	{\begin {proof} ~\\} 
	{\end {proof}}

\theoremstyle {definition}

\newtheorem {definition} {Définition} [subsection]
\newtheorem {lemme} {Lemme} [subsection]
\newtheorem {theoreme} {Théorème} [subsection]

%============================================================

\title {
	Recherche de fonctions par types \\[0.5em]
	\large Un peu de théorie
}
\date {}

%============================================================

\newcommand {\interval} [2] {[\![#1\,;#2]\!]}

\newcommand {\ssi} {\textit {ssi}}

\newcommand {\V} {\mathscr V}
\newcommand {\C} {\mathscr C}
\newcommand {\T} {\mathscr T}
\newcommand {\E} {\mathscr E}

\newcommand {\qeq} {\stackrel ? =}
\newcommand {\Eeq} {\stackrel \E =}
\newcommand {\ueq} {\stackrel u =}

%============================================================
%============================================================

\begin {document}

\maketitle

%============================================================
%============================================================

\section {Cadre général}

\subsection {Variables}

Dans toute la suite, on se donne un ensemble dénombrable $\V$ de symboles de variables. On signifiera par $v$ une variable dans les définitions inductives.

En Coq, cet ensemble est défini dans \texttt {Var.v}.

\subsection {Signature}

\begin {definition} [signature]
	Une signature est un ensemble $\Sigma$ de symboles muni d'une fonction d'arité $| \cdot |_\Sigma$ de $\Sigma$ dans $\mathbb N$.
\end {definition}

Dans toute la suite, on se donne une signature $( \Sigma, | \cdot |_\Sigma )$ telle que $\V \cap \Sigma = \emptyset$. On signifiera par $f$ ou $g$ un symbole de $\Sigma$ dans les définitions inductives.

Le fichier Coq correspondant est \texttt {Signature.v}.

%============================================================
%============================================================

\section {Etude sans $unit$}

En Coq, cette section correspond au fichier \texttt {TypWithoutUnit.v}.

%============================================================

\subsection {Types}

\begin {definition} [type]
	L'ensemble des types, noté $T$, est défini inductivement par :
	\begin {mathpar}
	    \inferrule* 
	    	{ }
	    	{\V \subseteq T}
	    \and
	    \inferrule*
	    	{ }
	    	{unit \in T}
	    \\
	    \inferrule*
	    	{\tau_1 \in T \\ \tau_2 \in T}
	    	{\tau_1 * \tau_2 \in T}
	    \and
	   	\inferrule*
	   		{\tau_1 \in T \\ \tau_2 \in T}
	   		{\tau_1 \rightarrow \tau_2 \in T}
	  	\\
	    \inferrule*
	    	{f \in \Sigma \\ \forall i \in \interval 1 {|f|_\Sigma},\ \tau_i \in T}
	    	{f (\tau_1, \dots, \tau_{|f|_\Sigma}) \in T}
	\end {mathpar}
\end {definition}

\begin {definition} [type-flèche]
	Un type-flèche (ou une flèche) est un type de la forme : $\tau_1 \rightarrow \tau_2$.
\end {definition}

\begin {definition} [racine d'un type]
	La racine d'un type, notée $[ \cdot ]$, est le symbole défini inductivement par :
	\begin {align*}
		[ v ] &= v \\
		[ unit ] &= unit \\
		[ \tau_1 * \tau_2 ] &= * \\
		[ \tau_1 \rightarrow \tau_2 \in T ] &=\ \rightarrow \\
		[ f (\tau_1, \dots, \tau_{|f|_\Sigma}) ] &= f
	\end {align*}
\end {definition}

\begin {definition} [substitution de types]
	Une substitution de types est une fonction de $\V$ dans $T$.
\end {definition}

\begin {definition} [extension d'une substitution de types]
	Soit $\sigma$ une substitution de types. \\
	L'extension de $\sigma$, notée $\hat \sigma$, est définie inductivement par :
	\begin {align*}
		\hat \sigma (v) &= \sigma (v) \\
		\hat \sigma (unit) &= unit \\
		\hat \sigma (\tau_1 * \tau_2) &= \hat \sigma (\tau_1) * \hat \sigma (\tau_2) \\
		\hat \sigma (\tau_1 \rightarrow \tau_2) &= \hat \sigma (\tau_1) \rightarrow \hat \sigma (\tau_2) \\
		\hat \sigma (f (\tau_1, \dots, \tau_{|f|_\Sigma})) &= f (\hat \sigma (\tau_1), \dots, \hat \sigma (\tau_{|f|_\Sigma}))
	\end {align*}
\end {definition}

\begin {definition} [instance de type]
	Un type $\tau_1$ est une instance d'un type $\tau_2$, noté $\tau_2 \leqslant_T \tau_1$, s'il existe une substitution de types $\sigma$  telle que $\tau_1 = \hat \sigma (\tau_2)$.
\end {definition}

%============================================================

\subsection {Théorie équationnelle}

\begin {definition} [axiome équationnel]
	Un axiome équationnel est un couple de types de la forme $\tau_1 \doteq \tau_2$.
\end {definition}

\begin {definition} [instance d'axiome équationnel]
	Une instance d'un axiome équationnel $\tau_1 \doteq \tau_2$ est un couple de types $(\tau_1', \tau_2')$ tels que $\tau_1'$ soit une instance de $\tau_1$ et $\tau_2'$ une instance de $\tau_2$.
\end {definition}

\begin {definition} [théorie équationnelle]
	Soit $\E$ un ensemble d'axiomes équationnels. \\
	La théorie équationnelle induite par $\E$, notée $\cdot \Eeq \cdot$, est la plus petite congruence sur $\Sigma$ contenant toutes les instances des axiomes équationnels de $\E$. \\
	Autrement dit, c'est la plus petite relation binaire satisfaisant les règles d'inférence :
	\begin {mathpar}
		\inferrule*
			[right = ($\Eeq$-ax)]
			{\tau_1 \doteq \tau_2 \in \E}
			{\hat \sigma (\tau_1) \Eeq \hat \sigma (\tau_2)}
		\and
		\inferrule*
			[right = ($\Eeq$-refl)]
			{ }
			{\tau \Eeq \tau}
		\\
		\inferrule*
			[right = ($\Eeq$-trans)]
			{\tau_1 \Eeq \tau_2 \\ \tau_2 \Eeq \tau_3}
			{\tau_1 \Eeq \tau_3}
		\and
		\inferrule*
			[right = ($\Eeq$-sym)]
			{\tau_1 \Eeq \tau_2}
			{\tau_2 \Eeq \tau_1}
		\\
		\inferrule*
			[right = ($\Eeq$-cong-$*_1$)]
			{\tau_1 \Eeq \tau_1'}
			{\tau_1 * \tau_2 \Eeq \tau_1' * \tau_2}
		\and
		\inferrule*
			[right = ($\Eeq$-cong-$*_2$)]
			{\tau_2 \Eeq \tau_2'}
			{\tau_1 * \tau_2 \Eeq \tau_1 * \tau_2'}
		\\
		\inferrule*
			[right = ($\Eeq$-cong-$\rightarrow_1$)]
			{\tau_1 \Eeq \tau_1'}
			{\tau_1 \rightarrow \tau_2 \Eeq \tau_1' \rightarrow \tau_2}
		\and
		\inferrule*
			[right = ($\Eeq$-cong-$\rightarrow_2$)]
			{\tau_2 \Eeq \tau_2'}
			{\tau_1 \rightarrow \tau_2 \Eeq \tau_1 \rightarrow \tau_2'}
		\\
		\inferrule*
			[right = ($\Eeq$-cong-$\Sigma$)]
			{f \in \Sigma \\ i \in \interval 1 {|f|_\Sigma} \\ \tau_i \Eeq \tau_i'}
			{f (\tau_1, \dots, \tau_i, \dots, \tau_{|f|_\Sigma}) \Eeq f (\tau_1, \dots, \tau_i', \dots, \tau_{|f|_\Sigma})}
	\end {mathpar}
\end {definition}

Dans toute la suite, on s'intéressa à l'ensemble $\E$ des axiomes équationnels (où $x$, $y$ et $z$ désignent des variables) :
\begin {align*}
	x * (y * z) &\doteq (x * y) * z && \text {($*$-assoc)} \\
	x * y &\doteq y * x && \text {($*$-comm)} \\
	x * y \rightarrow z &\doteq x \rightarrow y \rightarrow z && \text {(curry)}
\end {align*}

On considérera également la théorie équationnelle $\cdot \Eeq \cdot$ induite par $\E$.

\begin {definition} [équivalence]
	Deux types $\tau_1$ et $\tau_2$ sont équivalents {\ssi} $\tau_1 \Eeq \tau_2$.
\end {definition}

\begin {definition} [unifiabilité]
	Deux types $\tau_1$ et $\tau_2$ sont unifiables, noté $\tau_1 \ueq \tau_2$, \ssi :
	\[ \exists \sigma \in \mathscr F (\mathscr V, T),\ \hat \sigma (\tau_1) \Eeq \hat \sigma (\tau_2) \]
\end {definition}

%============================================================
%============================================================

\section {Premier critère (tête)}

\begin {definition} [tête d'un type]
	La tête d'un type, notée $\uparrow \cdot$, est définie inductivement par :
	\begin {align*}
		\uparrow (\tau_1 \rightarrow \tau_2) &=\ \uparrow \tau_2 \\
		\uparrow \tau &= \tau
	\end {align*}
\end {definition}

\begin {lemme} \label {=E-tete}
	Si deux types sont équivalents, leurs têtes le sont aussi.
\end {lemme}

\begin {lemme} \label {tete-subst-tete}
	$\forall \tau \in T,\ \forall \sigma \in \mathscr F (\mathscr V, T),\ \uparrow \hat \sigma (\tau) =\ \uparrow \hat \sigma (\uparrow \tau)$
\end {lemme}

\begin {lemme} \label {tete-non-fleche}
	La tête d'un type n'est pas une flèche.
\end {lemme}

\begin {lemme} \label {cons-=E}
	\begin {align*}
		\forall f_1 \in \Sigma &,\ \forall f_2 \in \Sigma, \\
		\forall (\tau^1_i)_{i \in \interval 1 {|f_1|_\Sigma}} \in T^{|f_1|_\Sigma} &,\ \forall (\tau^2_i)_{i \in \interval 1 {|f_2|_\Sigma}} \in T^{|f_2|_\Sigma}, \\
		f_1 (\tau^1_1, \dots, \tau^1_{|f_1|_\Sigma}) &= f_2 (\tau^2_1, \dots, \tau^2_{|f_2|_\Sigma}) \\
		\implies f_1 &= f_2
	\end {align*}
\end {lemme}

\begin {theoreme}
	Si deux types $\tau_1$ et $\tau_2$ sont unifiables et les racines de leurs têtes dans $\Sigma$, alors ces racines sont les mêmes :
	\begin {align*}
		\forall \tau_1 \in T &,\ \forall \tau_2 \in T, \\
		\tau_1 &\ueq \tau_2 \ \wedge \\
		[ \uparrow \tau_1 ] = f_1 \in \Sigma &\wedge [ \uparrow \tau_2 ] = f_2 \in \Sigma \\
		\implies f_1 &= f_2
	\end {align*}
\end {theoreme}

\begin {preuve}
	Par hypothèse, il existe une substitution de types $\sigma$ telle que :
	\[ \hat \sigma (\tau_1) \Eeq \hat \sigma (\tau_2) \]
	Par le lemme \ref {=E-tete}, les têtes sont équivalentes :
	\[ \uparrow \hat \sigma (\tau_1) \Eeq\ \uparrow \hat \sigma (\tau_2) \]
	Par le lemme \ref {tete-subst-tete}, on a donc :
	\[ \uparrow \hat \sigma (\uparrow \tau_1) \Eeq\ \uparrow \hat \sigma (\uparrow \tau_2) \]
	Par hypothèse, il existe $f_1$ et $f_2$ dans $\Sigma$ ainsi que $(\tau^1_i)_{i \in \interval 1 {|f_1|_\Sigma}}$ dans $T^{|f_1|_\Sigma}$ et $(\tau^2_i)_{i \in \interval 1 {|f_2|_\Sigma}}$ dans $T^{|f_2|_\Sigma}$ tels que :
	\begin {align*}
		\tau_1 &= f_1 (\tau^1_1, \dots \tau^1_{|f_1|_\Sigma}) \\
		\tau_2 &= f_2 (\tau^2_1, \dots \tau^2_{|f_2|_\Sigma})
	\end {align*}
	Par le lemme \ref {tete-non-fleche}, $f_1$ et $f_2$ sont différents de $\rightarrow$. \\
	Par definition de $\uparrow \cdot$ et $\hat \sigma$, il vient alors :
	\[ f_1 (\hat \sigma (\tau^1_1), \dots, \hat \sigma (\tau^1_{|f_1|_\Sigma})) \Eeq f_2 (\hat \sigma (\tau^2_1), \dots, \hat \sigma (\tau^2_{|f_2|_\Sigma})) \]
	Enfin, le lemme \ref {cons-=E} donne :
	\[ f_1 = f_2 \]
\end {preuve}

%============================================================

\section {Deuxième critère (queue)}

\begin {definition} [multiplicité de symbole de fonctions]
	La multiplicité d'un symbole $f$ de $\Sigma$, notée $\mu_f$, est définie inductivement par :
	\begin {align*}
		\mu_f (\tau_1 \rightarrow \tau_2) &= \mu_f' (\tau_1) + \mu_f (\tau_2) \\
		\mu_f (\tau) &= 0 \\
		\mu_f' (\tau_1 * \tau_2) &= \mu_f' (\tau_1) + \mu_f' (\tau_2) \\
		\mu_f' (f (\tau_1, \dots, \tau_n)) &= 1 \\
		\mu_f' (\tau) &= 0
	\end {align*}
\end {definition}

\begin {definition} [$\V$-multiplicité]
	La $\V$-multiplicité est définie inductivement par :
	\begin {align*}
		\mu_\V (\tau_1 \rightarrow \tau_2) &= \mu_\V' (\tau_1) + \mu_\V (\tau_2) \\
		\mu_\V (\tau) &= 0 \\
		\mu_\V' (v) &= 1 \\
		\mu_\V' (\tau_1 * \tau_2) &= \mu_\V' (\tau_1) + \mu_\V' (\tau_2) \\
		\mu_\V' (\tau) &= 0
	\end {align*}
\end {definition}

\begin {definition} [type simpe]
	Un type $\tau$ est simple si sa $\V$-multiplicité est nulle.
\end {definition}

\begin {lemme} \label {mu-=E}
	Si deux type $\tau_1$ et $\tau_2$ sont équivalents, alors, pour tout symbole $f$ de $\Sigma$, on a : $\mu_f (\tau_1) = \mu_f (\tau_2)$.
\end {lemme}

\begin {lemme} \label {mu-subst-simple}
	Si un type $\tau$ est simple, alors, pour tout symbole $f$ de $\Sigma$ et toute substitution de types $\sigma$, on a : $\mu_f (\hat \sigma (\tau)) = \mu_f (\tau)$.
\end {lemme}

\begin {lemme} \label {mu-subst}
	La multiplicité de tout symbole $f$ de $\Sigma$ dans un type est inférieure à celle de toute instance de ce type.
\end {lemme}

\begin {theoreme}
	Soit deux types $\tau_1$ et $\tau_2$. \\
	Si $\tau_1$ et $\tau_2$ sont unifiables et $\tau_1$ simple, alors la multiplicité de tout symbole de $\Sigma$ dans $\tau_1$ est supérieure à celle dans $\tau_2$.
\end {theoreme}

\begin {preuve}
	Par hypothèse, il existe une substitution $\sigma$ telle que :
	\[ \hat \sigma (\tau_1) \Eeq \hat \sigma (\tau_2) \]
	Soit $f$ un symbole de $\Sigma$. \\
	Par le lemme \ref {mu-=E}, les multiplicités sont égales :
	\[ \mu_f (\hat \sigma (\tau_1)) = \mu_f (\hat \sigma (\tau_2)) \]
	Par le lemme \ref {mu-subst-simple}, la simplicité de $\tau_1$ apporte :
	\[ \mu_f (\hat \sigma (\tau_1)) = \mu_f (\tau_1) \]
	Par le lemme \ref {mu-subst}, on a par ailleurs :
	\[ \mu_f (\hat \sigma (\tau_2)) \geqslant \mu_f (\tau_2) \]
	Il vient donc le résultat attendu :
	\[ \mu_f (\tau_1) \geqslant \mu_f (\tau_2) \]
\end {preuve}

%============================================================
%============================================================

\end {document}

































